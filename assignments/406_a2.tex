
% Default to the notebook output style

    


% Inherit from the specified cell style.




    
\documentclass[11pt]{article}

    
    \usepackage[T1]{fontenc}
    % Nicer default font (+ math font) than Computer Modern for most use cases
    \usepackage{mathpazo}

    % Basic figure setup, for now with no caption control since it's done
    % automatically by Pandoc (which extracts ![](path) syntax from Markdown).
    \usepackage{graphicx}
    % We will generate all images so they have a width \maxwidth. This means
    % that they will get their normal width if they fit onto the page, but
    % are scaled down if they would overflow the margins.
%     \makeatletter
%     \def\maxwidth{\ifdim\Gin@nat@width>\linewidth\linewidth
%     \else\Gin@nat@width\fi}
%     \makeatother
%     \let\Oldincludegraphics\includegraphics
%     % Set max figure width to be 80% of text width, for now hardcoded.
%     \renewcommand{\includegraphics}[1]{\Oldincludegraphics[width=.8\maxwidth]{#1}}
    % Ensure that by default, figures have no caption (until we provide a
    % proper Figure object with a Caption API and a way to capture that
    % in the conversion process - todo).
    \usepackage{caption}
    \DeclareCaptionLabelFormat{nolabel}{}
    \captionsetup{labelformat=nolabel}

    \usepackage{adjustbox} % Used to constrain images to a maximum size 
    \usepackage{xcolor} % Allow colors to be defined
    \usepackage{enumerate} % Needed for markdown enumerations to work
    \usepackage{geometry} % Used to adjust the document margins
    \usepackage{amsmath} % Equations
    \usepackage{amssymb} % Equations
    \usepackage{textcomp} % defines textquotesingle
    % Hack from http://tex.stackexchange.com/a/47451/13684:
    \AtBeginDocument{%
        \def\PYZsq{\textquotesingle}% Upright quotes in Pygmentized code
    }
    \usepackage{upquote} % Upright quotes for verbatim code
    \usepackage{eurosym} % defines \euro
    \usepackage[mathletters]{ucs} % Extended unicode (utf-8) support
    \usepackage[utf8x]{inputenc} % Allow utf-8 characters in the tex document
    \usepackage{fancyvrb} % verbatim replacement that allows latex
    \usepackage{grffile} % extends the file name processing of package graphics 
                         % to support a larger range 
    % The hyperref package gives us a pdf with properly built
    % internal navigation ('pdf bookmarks' for the table of contents,
    % internal cross-reference links, web links for URLs, etc.)
    \usepackage{hyperref}
    \usepackage{longtable} % longtable support required by pandoc >1.10
    \usepackage{booktabs}  % table support for pandoc > 1.12.2
    \usepackage[inline]{enumitem} % IRkernel/repr support (it uses the enumerate* environment)
    \usepackage[normalem]{ulem} % ulem is needed to support strikethroughs (\sout)
                                % normalem makes italics be italics, not underlines
    

    
    
    % Colors for the hyperref package
    \definecolor{urlcolor}{rgb}{0,.145,.698}
    \definecolor{linkcolor}{rgb}{.71,0.21,0.01}
    \definecolor{citecolor}{rgb}{.12,.54,.11}

    % ANSI colors
    \definecolor{ansi-black}{HTML}{3E424D}
    \definecolor{ansi-black-intense}{HTML}{282C36}
    \definecolor{ansi-red}{HTML}{E75C58}
    \definecolor{ansi-red-intense}{HTML}{B22B31}
    \definecolor{ansi-green}{HTML}{00A250}
    \definecolor{ansi-green-intense}{HTML}{007427}
    \definecolor{ansi-yellow}{HTML}{DDB62B}
    \definecolor{ansi-yellow-intense}{HTML}{B27D12}
    \definecolor{ansi-blue}{HTML}{208FFB}
    \definecolor{ansi-blue-intense}{HTML}{0065CA}
    \definecolor{ansi-magenta}{HTML}{D160C4}
    \definecolor{ansi-magenta-intense}{HTML}{A03196}
    \definecolor{ansi-cyan}{HTML}{60C6C8}
    \definecolor{ansi-cyan-intense}{HTML}{258F8F}
    \definecolor{ansi-white}{HTML}{C5C1B4}
    \definecolor{ansi-white-intense}{HTML}{A1A6B2}

    % commands and environments needed by pandoc snippets
    % extracted from the output of `pandoc -s`
    \providecommand{\tightlist}{%
      \setlength{\itemsep}{0pt}\setlength{\parskip}{0pt}}
    \DefineVerbatimEnvironment{Highlighting}{Verbatim}{commandchars=\\\{\}}
    % Add ',fontsize=\small' for more characters per line
    \newenvironment{Shaded}{}{}
    \newcommand{\KeywordTok}[1]{\textcolor[rgb]{0.00,0.44,0.13}{\textbf{{#1}}}}
    \newcommand{\DataTypeTok}[1]{\textcolor[rgb]{0.56,0.13,0.00}{{#1}}}
    \newcommand{\DecValTok}[1]{\textcolor[rgb]{0.25,0.63,0.44}{{#1}}}
    \newcommand{\BaseNTok}[1]{\textcolor[rgb]{0.25,0.63,0.44}{{#1}}}
    \newcommand{\FloatTok}[1]{\textcolor[rgb]{0.25,0.63,0.44}{{#1}}}
    \newcommand{\CharTok}[1]{\textcolor[rgb]{0.25,0.44,0.63}{{#1}}}
    \newcommand{\StringTok}[1]{\textcolor[rgb]{0.25,0.44,0.63}{{#1}}}
    \newcommand{\CommentTok}[1]{\textcolor[rgb]{0.38,0.63,0.69}{\textit{{#1}}}}
    \newcommand{\OtherTok}[1]{\textcolor[rgb]{0.00,0.44,0.13}{{#1}}}
    \newcommand{\AlertTok}[1]{\textcolor[rgb]{1.00,0.00,0.00}{\textbf{{#1}}}}
    \newcommand{\FunctionTok}[1]{\textcolor[rgb]{0.02,0.16,0.49}{{#1}}}
    \newcommand{\RegionMarkerTok}[1]{{#1}}
    \newcommand{\ErrorTok}[1]{\textcolor[rgb]{1.00,0.00,0.00}{\textbf{{#1}}}}
    \newcommand{\NormalTok}[1]{{#1}}
    
    % Additional commands for more recent versions of Pandoc
    \newcommand{\ConstantTok}[1]{\textcolor[rgb]{0.53,0.00,0.00}{{#1}}}
    \newcommand{\SpecialCharTok}[1]{\textcolor[rgb]{0.25,0.44,0.63}{{#1}}}
    \newcommand{\VerbatimStringTok}[1]{\textcolor[rgb]{0.25,0.44,0.63}{{#1}}}
    \newcommand{\SpecialStringTok}[1]{\textcolor[rgb]{0.73,0.40,0.53}{{#1}}}
    \newcommand{\ImportTok}[1]{{#1}}
    \newcommand{\DocumentationTok}[1]{\textcolor[rgb]{0.73,0.13,0.13}{\textit{{#1}}}}
    \newcommand{\AnnotationTok}[1]{\textcolor[rgb]{0.38,0.63,0.69}{\textbf{\textit{{#1}}}}}
    \newcommand{\CommentVarTok}[1]{\textcolor[rgb]{0.38,0.63,0.69}{\textbf{\textit{{#1}}}}}
    \newcommand{\VariableTok}[1]{\textcolor[rgb]{0.10,0.09,0.49}{{#1}}}
    \newcommand{\ControlFlowTok}[1]{\textcolor[rgb]{0.00,0.44,0.13}{\textbf{{#1}}}}
    \newcommand{\OperatorTok}[1]{\textcolor[rgb]{0.40,0.40,0.40}{{#1}}}
    \newcommand{\BuiltInTok}[1]{{#1}}
    \newcommand{\ExtensionTok}[1]{{#1}}
    \newcommand{\PreprocessorTok}[1]{\textcolor[rgb]{0.74,0.48,0.00}{{#1}}}
    \newcommand{\AttributeTok}[1]{\textcolor[rgb]{0.49,0.56,0.16}{{#1}}}
    \newcommand{\InformationTok}[1]{\textcolor[rgb]{0.38,0.63,0.69}{\textbf{\textit{{#1}}}}}
    \newcommand{\WarningTok}[1]{\textcolor[rgb]{0.38,0.63,0.69}{\textbf{\textit{{#1}}}}}
    
    
    % Define a nice break command that doesn't care if a line doesn't already
    % exist.
    \def\br{\hspace*{\fill} \\* }
    % Math Jax compatability definitions
    \def\gt{>}
    \def\lt{<}
    % Document parameters
    \title{406\_a2}
    
    
    

    % Pygments definitions
    
\makeatletter
\def\PY@reset{\let\PY@it=\relax \let\PY@bf=\relax%
    \let\PY@ul=\relax \let\PY@tc=\relax%
    \let\PY@bc=\relax \let\PY@ff=\relax}
\def\PY@tok#1{\csname PY@tok@#1\endcsname}
\def\PY@toks#1+{\ifx\relax#1\empty\else%
    \PY@tok{#1}\expandafter\PY@toks\fi}
\def\PY@do#1{\PY@bc{\PY@tc{\PY@ul{%
    \PY@it{\PY@bf{\PY@ff{#1}}}}}}}
\def\PY#1#2{\PY@reset\PY@toks#1+\relax+\PY@do{#2}}

\expandafter\def\csname PY@tok@w\endcsname{\def\PY@tc##1{\textcolor[rgb]{0.73,0.73,0.73}{##1}}}
\expandafter\def\csname PY@tok@c\endcsname{\let\PY@it=\textit\def\PY@tc##1{\textcolor[rgb]{0.25,0.50,0.50}{##1}}}
\expandafter\def\csname PY@tok@cp\endcsname{\def\PY@tc##1{\textcolor[rgb]{0.74,0.48,0.00}{##1}}}
\expandafter\def\csname PY@tok@k\endcsname{\let\PY@bf=\textbf\def\PY@tc##1{\textcolor[rgb]{0.00,0.50,0.00}{##1}}}
\expandafter\def\csname PY@tok@kp\endcsname{\def\PY@tc##1{\textcolor[rgb]{0.00,0.50,0.00}{##1}}}
\expandafter\def\csname PY@tok@kt\endcsname{\def\PY@tc##1{\textcolor[rgb]{0.69,0.00,0.25}{##1}}}
\expandafter\def\csname PY@tok@o\endcsname{\def\PY@tc##1{\textcolor[rgb]{0.40,0.40,0.40}{##1}}}
\expandafter\def\csname PY@tok@ow\endcsname{\let\PY@bf=\textbf\def\PY@tc##1{\textcolor[rgb]{0.67,0.13,1.00}{##1}}}
\expandafter\def\csname PY@tok@nb\endcsname{\def\PY@tc##1{\textcolor[rgb]{0.00,0.50,0.00}{##1}}}
\expandafter\def\csname PY@tok@nf\endcsname{\def\PY@tc##1{\textcolor[rgb]{0.00,0.00,1.00}{##1}}}
\expandafter\def\csname PY@tok@nc\endcsname{\let\PY@bf=\textbf\def\PY@tc##1{\textcolor[rgb]{0.00,0.00,1.00}{##1}}}
\expandafter\def\csname PY@tok@nn\endcsname{\let\PY@bf=\textbf\def\PY@tc##1{\textcolor[rgb]{0.00,0.00,1.00}{##1}}}
\expandafter\def\csname PY@tok@ne\endcsname{\let\PY@bf=\textbf\def\PY@tc##1{\textcolor[rgb]{0.82,0.25,0.23}{##1}}}
\expandafter\def\csname PY@tok@nv\endcsname{\def\PY@tc##1{\textcolor[rgb]{0.10,0.09,0.49}{##1}}}
\expandafter\def\csname PY@tok@no\endcsname{\def\PY@tc##1{\textcolor[rgb]{0.53,0.00,0.00}{##1}}}
\expandafter\def\csname PY@tok@nl\endcsname{\def\PY@tc##1{\textcolor[rgb]{0.63,0.63,0.00}{##1}}}
\expandafter\def\csname PY@tok@ni\endcsname{\let\PY@bf=\textbf\def\PY@tc##1{\textcolor[rgb]{0.60,0.60,0.60}{##1}}}
\expandafter\def\csname PY@tok@na\endcsname{\def\PY@tc##1{\textcolor[rgb]{0.49,0.56,0.16}{##1}}}
\expandafter\def\csname PY@tok@nt\endcsname{\let\PY@bf=\textbf\def\PY@tc##1{\textcolor[rgb]{0.00,0.50,0.00}{##1}}}
\expandafter\def\csname PY@tok@nd\endcsname{\def\PY@tc##1{\textcolor[rgb]{0.67,0.13,1.00}{##1}}}
\expandafter\def\csname PY@tok@s\endcsname{\def\PY@tc##1{\textcolor[rgb]{0.73,0.13,0.13}{##1}}}
\expandafter\def\csname PY@tok@sd\endcsname{\let\PY@it=\textit\def\PY@tc##1{\textcolor[rgb]{0.73,0.13,0.13}{##1}}}
\expandafter\def\csname PY@tok@si\endcsname{\let\PY@bf=\textbf\def\PY@tc##1{\textcolor[rgb]{0.73,0.40,0.53}{##1}}}
\expandafter\def\csname PY@tok@se\endcsname{\let\PY@bf=\textbf\def\PY@tc##1{\textcolor[rgb]{0.73,0.40,0.13}{##1}}}
\expandafter\def\csname PY@tok@sr\endcsname{\def\PY@tc##1{\textcolor[rgb]{0.73,0.40,0.53}{##1}}}
\expandafter\def\csname PY@tok@ss\endcsname{\def\PY@tc##1{\textcolor[rgb]{0.10,0.09,0.49}{##1}}}
\expandafter\def\csname PY@tok@sx\endcsname{\def\PY@tc##1{\textcolor[rgb]{0.00,0.50,0.00}{##1}}}
\expandafter\def\csname PY@tok@m\endcsname{\def\PY@tc##1{\textcolor[rgb]{0.40,0.40,0.40}{##1}}}
\expandafter\def\csname PY@tok@gh\endcsname{\let\PY@bf=\textbf\def\PY@tc##1{\textcolor[rgb]{0.00,0.00,0.50}{##1}}}
\expandafter\def\csname PY@tok@gu\endcsname{\let\PY@bf=\textbf\def\PY@tc##1{\textcolor[rgb]{0.50,0.00,0.50}{##1}}}
\expandafter\def\csname PY@tok@gd\endcsname{\def\PY@tc##1{\textcolor[rgb]{0.63,0.00,0.00}{##1}}}
\expandafter\def\csname PY@tok@gi\endcsname{\def\PY@tc##1{\textcolor[rgb]{0.00,0.63,0.00}{##1}}}
\expandafter\def\csname PY@tok@gr\endcsname{\def\PY@tc##1{\textcolor[rgb]{1.00,0.00,0.00}{##1}}}
\expandafter\def\csname PY@tok@ge\endcsname{\let\PY@it=\textit}
\expandafter\def\csname PY@tok@gs\endcsname{\let\PY@bf=\textbf}
\expandafter\def\csname PY@tok@gp\endcsname{\let\PY@bf=\textbf\def\PY@tc##1{\textcolor[rgb]{0.00,0.00,0.50}{##1}}}
\expandafter\def\csname PY@tok@go\endcsname{\def\PY@tc##1{\textcolor[rgb]{0.53,0.53,0.53}{##1}}}
\expandafter\def\csname PY@tok@gt\endcsname{\def\PY@tc##1{\textcolor[rgb]{0.00,0.27,0.87}{##1}}}
\expandafter\def\csname PY@tok@err\endcsname{\def\PY@bc##1{\setlength{\fboxsep}{0pt}\fcolorbox[rgb]{1.00,0.00,0.00}{1,1,1}{\strut ##1}}}
\expandafter\def\csname PY@tok@kc\endcsname{\let\PY@bf=\textbf\def\PY@tc##1{\textcolor[rgb]{0.00,0.50,0.00}{##1}}}
\expandafter\def\csname PY@tok@kd\endcsname{\let\PY@bf=\textbf\def\PY@tc##1{\textcolor[rgb]{0.00,0.50,0.00}{##1}}}
\expandafter\def\csname PY@tok@kn\endcsname{\let\PY@bf=\textbf\def\PY@tc##1{\textcolor[rgb]{0.00,0.50,0.00}{##1}}}
\expandafter\def\csname PY@tok@kr\endcsname{\let\PY@bf=\textbf\def\PY@tc##1{\textcolor[rgb]{0.00,0.50,0.00}{##1}}}
\expandafter\def\csname PY@tok@bp\endcsname{\def\PY@tc##1{\textcolor[rgb]{0.00,0.50,0.00}{##1}}}
\expandafter\def\csname PY@tok@fm\endcsname{\def\PY@tc##1{\textcolor[rgb]{0.00,0.00,1.00}{##1}}}
\expandafter\def\csname PY@tok@vc\endcsname{\def\PY@tc##1{\textcolor[rgb]{0.10,0.09,0.49}{##1}}}
\expandafter\def\csname PY@tok@vg\endcsname{\def\PY@tc##1{\textcolor[rgb]{0.10,0.09,0.49}{##1}}}
\expandafter\def\csname PY@tok@vi\endcsname{\def\PY@tc##1{\textcolor[rgb]{0.10,0.09,0.49}{##1}}}
\expandafter\def\csname PY@tok@vm\endcsname{\def\PY@tc##1{\textcolor[rgb]{0.10,0.09,0.49}{##1}}}
\expandafter\def\csname PY@tok@sa\endcsname{\def\PY@tc##1{\textcolor[rgb]{0.73,0.13,0.13}{##1}}}
\expandafter\def\csname PY@tok@sb\endcsname{\def\PY@tc##1{\textcolor[rgb]{0.73,0.13,0.13}{##1}}}
\expandafter\def\csname PY@tok@sc\endcsname{\def\PY@tc##1{\textcolor[rgb]{0.73,0.13,0.13}{##1}}}
\expandafter\def\csname PY@tok@dl\endcsname{\def\PY@tc##1{\textcolor[rgb]{0.73,0.13,0.13}{##1}}}
\expandafter\def\csname PY@tok@s2\endcsname{\def\PY@tc##1{\textcolor[rgb]{0.73,0.13,0.13}{##1}}}
\expandafter\def\csname PY@tok@sh\endcsname{\def\PY@tc##1{\textcolor[rgb]{0.73,0.13,0.13}{##1}}}
\expandafter\def\csname PY@tok@s1\endcsname{\def\PY@tc##1{\textcolor[rgb]{0.73,0.13,0.13}{##1}}}
\expandafter\def\csname PY@tok@mb\endcsname{\def\PY@tc##1{\textcolor[rgb]{0.40,0.40,0.40}{##1}}}
\expandafter\def\csname PY@tok@mf\endcsname{\def\PY@tc##1{\textcolor[rgb]{0.40,0.40,0.40}{##1}}}
\expandafter\def\csname PY@tok@mh\endcsname{\def\PY@tc##1{\textcolor[rgb]{0.40,0.40,0.40}{##1}}}
\expandafter\def\csname PY@tok@mi\endcsname{\def\PY@tc##1{\textcolor[rgb]{0.40,0.40,0.40}{##1}}}
\expandafter\def\csname PY@tok@il\endcsname{\def\PY@tc##1{\textcolor[rgb]{0.40,0.40,0.40}{##1}}}
\expandafter\def\csname PY@tok@mo\endcsname{\def\PY@tc##1{\textcolor[rgb]{0.40,0.40,0.40}{##1}}}
\expandafter\def\csname PY@tok@ch\endcsname{\let\PY@it=\textit\def\PY@tc##1{\textcolor[rgb]{0.25,0.50,0.50}{##1}}}
\expandafter\def\csname PY@tok@cm\endcsname{\let\PY@it=\textit\def\PY@tc##1{\textcolor[rgb]{0.25,0.50,0.50}{##1}}}
\expandafter\def\csname PY@tok@cpf\endcsname{\let\PY@it=\textit\def\PY@tc##1{\textcolor[rgb]{0.25,0.50,0.50}{##1}}}
\expandafter\def\csname PY@tok@c1\endcsname{\let\PY@it=\textit\def\PY@tc##1{\textcolor[rgb]{0.25,0.50,0.50}{##1}}}
\expandafter\def\csname PY@tok@cs\endcsname{\let\PY@it=\textit\def\PY@tc##1{\textcolor[rgb]{0.25,0.50,0.50}{##1}}}

\def\PYZbs{\char`\\}
\def\PYZus{\char`\_}
\def\PYZob{\char`\{}
\def\PYZcb{\char`\}}
\def\PYZca{\char`\^}
\def\PYZam{\char`\&}
\def\PYZlt{\char`\<}
\def\PYZgt{\char`\>}
\def\PYZsh{\char`\#}
\def\PYZpc{\char`\%}
\def\PYZdl{\char`\$}
\def\PYZhy{\char`\-}
\def\PYZsq{\char`\'}
\def\PYZdq{\char`\"}
\def\PYZti{\char`\~}
% for compatibility with earlier versions
\def\PYZat{@}
\def\PYZlb{[}
\def\PYZrb{]}
\makeatother


    % Exact colors from NB
    \definecolor{incolor}{rgb}{0.0, 0.0, 0.5}
    \definecolor{outcolor}{rgb}{0.545, 0.0, 0.0}



    
    % Prevent overflowing lines due to hard-to-break entities
    \sloppy 
    % Setup hyperref package
    \hypersetup{
      breaklinks=true,  % so long urls are correctly broken across lines
      colorlinks=true,
      urlcolor=urlcolor,
      linkcolor=linkcolor,
      citecolor=citecolor,
      }
    % Slightly bigger margins than the latex defaults
    
    \geometry{verbose,tmargin=1in,bmargin=1in,lmargin=1in,rmargin=1in}
    
    

    \begin{document}
    
    
    \maketitle

    \hypertarget{exercise-1}{%
\section{Exercise 1}\label{exercise-1}}

    \hypertarget{a}{%
\subsubsection{a)}\label{a}}
    
    $A =\begin{bmatrix}
    1 & x_1^1       & x_2^1\\
    1 & x_1^2       & x_2^2\\
    & \vdots & \\
    1 &x_1^n    & x_2^n
\end{bmatrix},
\space 
x =  \begin{bmatrix}
    c_0\\
    c_1\\
    c_2
\end{bmatrix},
\space
b = \begin{bmatrix}
    b_1\\
    b_2\\
    \vdots\\
    b_n
\end{bmatrix}$

    

    \hypertarget{b}{%
\subsubsection{b)}\label{b}}

    \begin{Verbatim}[commandchars=\\\{\}]
{\color{incolor}In [{\color{incolor}4}]:} \PY{n}{data} \PY{o}{=} \PY{n}{CSV}\PY{o}{.}\PY{n}{read}\PY{p}{(}\PY{l+s}{\PYZdq{}}\PY{l+s}{C}\PY{l+s}{:}\PY{l+s+se}{\PYZbs{}\PYZbs{}}\PY{l+s}{U}\PY{l+s}{s}\PY{l+s}{e}\PY{l+s}{r}\PY{l+s}{s}\PY{l+s+se}{\PYZbs{}\PYZbs{}}\PY{l+s}{K}\PY{l+s}{y}\PY{l+s}{l}\PY{l+s}{e}\PY{l+s+se}{\PYZbs{}\PYZbs{}}\PY{l+s}{D}\PY{l+s}{e}\PY{l+s}{s}\PY{l+s}{k}\PY{l+s}{t}\PY{l+s}{o}\PY{l+s}{p}\PY{l+s+se}{\PYZbs{}\PYZbs{}}\PY{l+s}{l}\PY{l+s}{s}\PY{l+s}{q}\PY{l+s}{\PYZus{}}\PY{l+s}{c}\PY{l+s}{l}\PY{l+s}{a}\PY{l+s}{s}\PY{l+s}{s}\PY{l+s}{i}\PY{l+s}{f}\PY{l+s}{i}\PY{l+s}{c}\PY{l+s}{a}\PY{l+s}{t}\PY{l+s}{i}\PY{l+s}{o}\PY{l+s}{n}\PY{l+s}{.}\PY{l+s}{c}\PY{l+s}{s}\PY{l+s}{v}\PY{l+s}{\PYZdq{}}\PY{p}{;} \PY{n}{datarow}\PY{o}{=}\PY{l+m+mi}{1}\PY{p}{)}
        \PY{n}{mtx} \PY{o}{=} \PY{n}{convert}\PY{p}{(}\PY{k+kt}{Array}\PY{p}{,} \PY{n}{data}\PY{p}{)}
        \PY{n}{n} \PY{o}{=} \PY{n}{size}\PY{p}{(}\PY{n}{mtx}\PY{p}{,} \PY{l+m+mi}{1}\PY{p}{)}
        \PY{n}{c} \PY{o}{=} \PY{n}{mtx}\PY{p}{[}\PY{o}{:}\PY{p}{,}\PY{l+m+mi}{1}\PY{o}{:}\PY{l+m+mi}{2}\PY{p}{]}
        \PY{n}{A} \PY{o}{=} \PY{n}{hcat}\PY{p}{(}\PY{n}{ones}\PY{p}{(}\PY{n}{n}\PY{p}{,}\PY{l+m+mi}{1}\PY{p}{)}\PY{p}{,} \PY{n}{mtx}\PY{p}{[}\PY{o}{:}\PY{p}{,}\PY{l+m+mi}{1}\PY{o}{:}\PY{l+m+mi}{2}\PY{p}{]}\PY{p}{)}
        \PY{n}{betas} \PY{o}{=} \PY{n}{mtx}\PY{p}{[}\PY{o}{:}\PY{p}{,}\PY{l+m+mi}{3}\PY{o}{:}\PY{l+m+mi}{3}\PY{p}{]}
\end{Verbatim}


\begin{Verbatim}[commandchars=\\\{\}]
{\color{outcolor}Out[{\color{outcolor}4}]:} 115×1 Array\{Float64,2\}:
          1.0
          1.0
          1.0
          1.0
          1.0
          1.0
          1.0
          1.0
          1.0
          1.0
          1.0
          1.0
          1.0
          ⋮  
         -1.0
         -1.0
         -1.0
         -1.0
         -1.0
         -1.0
         -1.0
         -1.0
         -1.0
         -1.0
         -1.0
         -1.0
\end{Verbatim}
            
    \begin{verbatim}
i)
\end{verbatim}

    \begin{Verbatim}[commandchars=\\\{\}]
{\color{incolor}In [{\color{incolor}5}]:} \PY{c}{\PYZsh{} Solution via normal equation A\PYZca{}TAx = A\PYZca{}Tb}
        \PY{n}{xls} \PY{o}{=} \PY{o}{\PYZbs{}}\PY{p}{(}\PY{o}{*}\PY{p}{(}\PY{n}{transpose}\PY{p}{(}\PY{n}{A}\PY{p}{)}\PY{p}{,}\PY{n}{A}\PY{p}{)}\PY{p}{,} \PY{o}{*}\PY{p}{(}\PY{n}{transpose}\PY{p}{(}\PY{n}{A}\PY{p}{)}\PY{p}{,}\PY{n}{betas}\PY{p}{)}\PY{p}{)}
\end{Verbatim}


\begin{Verbatim}[commandchars=\\\{\}]
{\color{outcolor}Out[{\color{outcolor}5}]:} 3×1 Array\{Float64,2\}:
         -0.368611
          0.576988
          0.63353 
\end{Verbatim}
            
    \begin{verbatim}
ii)
\end{verbatim}

    \begin{Verbatim}[commandchars=\\\{\}]
{\color{incolor}In [{\color{incolor}6}]:} \PY{c}{\PYZsh{} Solution via QR}
        \PY{n}{QR} \PY{o}{=} \PY{n}{qrfact}\PY{p}{(}\PY{n}{A}\PY{p}{)}
        \PY{n}{Q} \PY{o}{=} \PY{n}{QR}\PY{p}{[}\PY{o}{:}\PY{n}{Q}\PY{p}{]}
        \PY{n}{R} \PY{o}{=} \PY{n}{QR}\PY{p}{[}\PY{o}{:}\PY{n}{R}\PY{p}{]}
        \PY{n}{m} \PY{o}{=} \PY{n}{size}\PY{p}{(}\PY{n}{R}\PY{p}{,}\PY{l+m+mi}{1}\PY{p}{)}
        
        \PY{n}{y} \PY{o}{=} \PY{o}{*}\PY{p}{(}\PY{n}{transpose}\PY{p}{(}\PY{n}{Q}\PY{p}{)}\PY{p}{,} \PY{n}{betas}\PY{p}{)}
        \PY{n}{y} \PY{o}{=} \PY{n}{y}\PY{p}{[}\PY{l+m+mi}{1}\PY{o}{:}\PY{n}{m}\PY{p}{]}
        \PY{n}{xqr} \PY{o}{=} \PY{o}{\PYZbs{}}\PY{p}{(}\PY{n}{R}\PY{p}{,} \PY{n}{y}\PY{p}{)}
\end{Verbatim}


\begin{Verbatim}[commandchars=\\\{\}]
{\color{outcolor}Out[{\color{outcolor}6}]:} 3-element Array\{Float64,1\}:
         -0.368611
          0.576988
          0.63353 
\end{Verbatim}
            
    \hypertarget{c}{%
\subsubsection{c)}\label{c}}

    \begin{Verbatim}[commandchars=\\\{\}]
{\color{incolor}In [{\color{incolor}7}]:} \PY{c}{\PYZsh{} define coefficients and function}
        \PY{n}{c0}\PY{p}{,} \PY{n}{c1}\PY{p}{,} \PY{n}{c2} \PY{o}{=} \PY{n}{reshape}\PY{p}{(}\PY{n}{xls}\PY{p}{[}\PY{l+m+mi}{1}\PY{o}{:}\PY{l+m+mi}{1}\PY{p}{]}\PY{p}{,}\PY{l+m+mi}{1}\PY{p}{)}\PY{p}{[}\PY{l+m+mi}{1}\PY{p}{]}\PY{p}{,} \PY{n}{reshape}\PY{p}{(}\PY{n}{xls}\PY{p}{[}\PY{l+m+mi}{2}\PY{o}{:}\PY{l+m+mi}{2}\PY{p}{]}\PY{p}{,}\PY{l+m+mi}{1}\PY{p}{)}\PY{p}{[}\PY{l+m+mi}{1}\PY{p}{]}\PY{p}{,} \PY{n}{reshape}\PY{p}{(}\PY{n}{xls}\PY{p}{[}\PY{l+m+mi}{3}\PY{o}{:}\PY{l+m+mi}{3}\PY{p}{]}\PY{p}{,}\PY{l+m+mi}{1}\PY{p}{)}\PY{p}{[}\PY{l+m+mi}{1}\PY{p}{]}
        \PY{n}{f}\PY{p}{(}\PY{n}{x1}\PY{p}{)} \PY{o}{=} \PY{p}{(}\PY{o}{\PYZhy{}}\PY{n}{c0} \PY{o}{\PYZhy{}}\PY{n}{c1}\PY{o}{*}\PY{n}{x1}\PY{p}{)} \PY{o}{/} \PY{n}{c2}
        
        \PY{c}{\PYZsh{} \PYZsh{} \PYZsh{} (x1,x2) where label = 1}
        \PY{n}{l11} \PY{o}{=} \PY{n}{mtx}\PY{p}{[}\PY{o}{:}\PY{p}{,}\PY{l+m+mi}{1}\PY{o}{:}\PY{l+m+mi}{1}\PY{p}{]}\PY{p}{[}\PY{n}{mtx}\PY{p}{[}\PY{o}{:}\PY{p}{,}\PY{l+m+mi}{3}\PY{o}{:}\PY{l+m+mi}{3}\PY{p}{]} \PY{o}{.==} \PY{l+m+mi}{1}\PY{p}{]}
        \PY{n}{l12} \PY{o}{=} \PY{n}{mtx}\PY{p}{[}\PY{o}{:}\PY{p}{,}\PY{l+m+mi}{2}\PY{o}{:}\PY{l+m+mi}{2}\PY{p}{]}\PY{p}{[}\PY{n}{mtx}\PY{p}{[}\PY{o}{:}\PY{p}{,}\PY{l+m+mi}{3}\PY{o}{:}\PY{l+m+mi}{3}\PY{p}{]} \PY{o}{.==} \PY{l+m+mi}{1}\PY{p}{]}
        
        \PY{c}{\PYZsh{} \PYZsh{} \PYZsh{} (x1,x2) where label = \PYZhy{}1}
        \PY{n}{l21} \PY{o}{=} \PY{n}{mtx}\PY{p}{[}\PY{o}{:}\PY{p}{,}\PY{l+m+mi}{1}\PY{o}{:}\PY{l+m+mi}{1}\PY{p}{]}\PY{p}{[}\PY{n}{mtx}\PY{p}{[}\PY{o}{:}\PY{p}{,}\PY{l+m+mi}{3}\PY{o}{:}\PY{l+m+mi}{3}\PY{p}{]} \PY{o}{.==} \PY{o}{\PYZhy{}}\PY{l+m+mi}{1}\PY{p}{]}
        \PY{n}{l22} \PY{o}{=} \PY{n}{mtx}\PY{p}{[}\PY{o}{:}\PY{p}{,}\PY{l+m+mi}{2}\PY{o}{:}\PY{l+m+mi}{2}\PY{p}{]}\PY{p}{[}\PY{n}{mtx}\PY{p}{[}\PY{o}{:}\PY{p}{,}\PY{l+m+mi}{3}\PY{o}{:}\PY{l+m+mi}{3}\PY{p}{]} \PY{o}{.==} \PY{o}{\PYZhy{}}\PY{l+m+mi}{1}\PY{p}{]}
        
        \PY{c}{\PYZsh{} \PYZsh{} \PYZsh{} plot data}
        \PY{n}{scatter}\PY{p}{(}\PY{n}{l11}\PY{p}{,} \PY{n}{l12}\PY{p}{,} \PY{n}{xlim}\PY{o}{=}\PY{p}{(}\PY{o}{\PYZhy{}}\PY{l+m+mf}{1.5}\PY{p}{,}\PY{l+m+mi}{2}\PY{p}{)}\PY{p}{,} \PY{n}{ylim}\PY{o}{=}\PY{p}{(}\PY{o}{\PYZhy{}}\PY{l+m+mi}{1}\PY{p}{,} \PY{l+m+mi}{3}\PY{p}{)}\PY{p}{,} \PY{n}{colour}\PY{o}{=}\PY{l+s}{\PYZdq{}}\PY{l+s}{b}\PY{l+s}{l}\PY{l+s}{u}\PY{l+s}{e}\PY{l+s}{\PYZdq{}}\PY{p}{)}
        \PY{n}{scatter!}\PY{p}{(}\PY{n}{l21}\PY{p}{,} \PY{n}{l22}\PY{p}{,} \PY{n}{colour}\PY{o}{=}\PY{l+s}{\PYZdq{}}\PY{l+s}{y}\PY{l+s}{e}\PY{l+s}{l}\PY{l+s}{l}\PY{l+s}{o}\PY{l+s}{w}\PY{l+s}{\PYZdq{}}\PY{p}{)}
        \PY{n}{plot!}\PY{p}{(}\PY{n}{f}\PY{p}{)}
\end{Verbatim}

\includegraphics[scale=.5]{a.png}
    
    
\hypertarget{exercise-2}{%
\section{Exercise 2}\label{exercise-2}}

\begin{verbatim}
1-norm
\end{verbatim}
    
$f(\alpha) = ||\alpha e - b|| = \sum_{i=1}^n|\alpha - b_i|$\\
This shows that by minimizing f(a) we are really just minimizing the sum
of the distances between a and the various values of b. To do this we
should let a = median(b)

\begin{verbatim}
2-norm
\end{verbatim}
    
$\min\;f(\alpha) = ||\alpha e - b|| \Rightarrow \min\; f(\alpha)^2 = \min\;\sum_{i=1}^n(\alpha - b_i)^2\\
\Rightarrow 2\alpha n - 2\sum_{i=1}^nb_i\\
\Rightarrow \alpha n = \sum_{i=1}^nb_i\\
\Rightarrow \alpha = \frac{1}{n}\sum_{i=1}^nb_i\\
\Rightarrow \alpha = mean(b)
$

    

    \begin{verbatim}
inf-norm
\end{verbatim}
    
$\min_\alpha f(\alpha) = min_\alpha\{max_i\{|\alpha - b_i|\}\} = min_\alpha\{max_i\{\alpha - b_i, b_i - \alpha\}\}$\\
To maximize $\alpha-b_i$ we would pick the smallest $b_i$. To maximize $b_i-\alpha$ we would pick the largest $b_i$.Therefore, to minimize both components we should let $\alpha = \frac{max{b_i} - min{b_i}}{2}$

    

    \hypertarget{exercise-3}{%
\section{Exercise 3}\label{exercise-3}}
    
    $f(x) = \frac{1}{2}(Ax-b)^TW(Ax-B) = \frac{1}{2}(Ax-b)^TQDQ(Ax-b)\\
= \frac{1}{2}(Ax-b)^TQ\sqrt[]{D}(\sqrt[]{D})^TQ(Ax-b)\\
Let\;Q\sqrt[]{D} = P\\
\Rightarrow \frac{1}{2}(Ax-b)^TPP^T(Ax-b)\\
= \frac{1}{2}(P^TAx-P^Tb)^T(P^TAx-P^Tb)\\
= \frac{1}{2}(Bx - d)^T(Bx - d)
$

    

    \hypertarget{exercise-4}{%
\section{Exercise 4}\label{exercise-4}}

    \begin{verbatim}
a) Since A is underdetermined and consistent, should pick x such that
\end{verbatim}
    
    $minimize\;||x||^2\\
subject\;to: Ax=b\\\\
let\; L(x,\mu) = ||x||^2 - \mu^T(b- Ax)\\
\frac{\partial}{\partial x} = 2x - u^TA\\\\
\Rightarrow x = \frac{1}{2}A^T\mu\\
\frac{\partial}{\partial \mu} = b - Ax\\
\Rightarrow b = Ax\\
\Rightarrow b = A(\frac{1}{2}A^T\mu)\\
\Rightarrow \mu = 2(AA^T)^{-1}b
\Rightarrow x = A^T(AA^T)^{-1}b
\\ Notice\;that\;A^T\;is\;overdetermined\;so\;let\; A^T = QR\\
\Rightarrow x = QR(R^TQ^TQR)^{-1}b = QR^{-T}b\\
We\;know\;the\;norm\;is\;invariant\;to\;orthogonal\;transformations\;therefore:\\
||x_{ls}|| = ||R^{-T}b||
$

    

    \begin{verbatim}
b) 
\end{verbatim}

    \begin{Verbatim}[commandchars=\\\{\}]
{\color{incolor}In [{\color{incolor}11}]:} \PY{n}{m} \PY{o}{=} \PY{l+m+mi}{5}
         \PY{n}{n} \PY{o}{=} \PY{l+m+mi}{10}
         \PY{n}{A} \PY{o}{=} \PY{n}{randn}\PY{p}{(}\PY{l+m+mi}{5}\PY{p}{,}\PY{l+m+mi}{10}\PY{p}{)}
         \PY{n}{b} \PY{o}{=} \PY{n}{randn}\PY{p}{(}\PY{l+m+mi}{5}\PY{p}{)}
         \PY{n}{At} \PY{o}{=} \PY{n}{A}\PY{o}{\PYZsq{}}
         \PY{n}{QR} \PY{o}{=}\PY{n}{qrfact}\PY{p}{(}\PY{n}{At}\PY{p}{)}
         \PY{n}{Q} \PY{o}{=} \PY{n}{QR}\PY{p}{[}\PY{o}{:}\PY{n}{Q}\PY{p}{]}
         \PY{n}{R} \PY{o}{=} \PY{n}{QR}\PY{p}{[}\PY{o}{:}\PY{n}{R}\PY{p}{]}
         \PY{n}{xls} \PY{o}{=} \PY{o}{*}\PY{p}{(}\PY{o}{*}\PY{p}{(}\PY{n}{Q}\PY{p}{,}\PY{n}{inv}\PY{p}{(}\PY{n}{R}\PY{o}{\PYZsq{}}\PY{p}{)}\PY{p}{)}\PY{p}{,}\PY{n}{b}\PY{p}{)}
\end{Verbatim}


\begin{Verbatim}[commandchars=\\\{\}]
{\color{outcolor}Out[{\color{outcolor}11}]:} 10-element Array\{Float64,1\}:
           1.01408   
           0.3956    
           0.187194  
           0.598118  
          -0.370701  
          -0.00259992
          -0.391118  
           0.245214  
           0.801711  
          -0.0776825 
\end{Verbatim}
            
    \hypertarget{exercise-5}{%
\section{Exercise 5}\label{exercise-5}}

    Proof: RLS has a unqiue solution, then null(A)\^{}null(L) = \{0\}.

    $f(x) = x^TA^TAx - 2x^TA^Tb + b^Tb + \lambda x^TL^TLx\\
\nabla f(x) = 2A^TAx - 2A^Tb + 2\lambda L^TLx \\
\Rightarrow A^TAx + \lambda L^TLx = A^Tb\\
\Rightarrow (A^TA + \lambda L^TL)x = A^Tb\\$
We know that $ (A^TA + \lambda L^TL)^{-1}$ must exist since RLS has a unique solution. So 0 is not an eigenvalue\\
$\Rightarrow y^T(A^TA + \lambda L^TL)y \Rightarrow y^TA^TAy + \lambda y^TL^TLy \Rightarrow ||Ay||^2 + \lambda||Ly||^2 > 0$\\
Since there is not y s.t the above equation is 0, it must be the case that the $null(A)\land null(L)= \{0\}$\\

Proof: $null(A)\land null(L) = \{0\}$ then RLS has a unqiue solution. Proof
by contradiciton, suppose $null(A)\land null(L) = \{0\}$ and RLS does not
have a unique solution\\

$Let\; x,y\in \mathbb R^n, x \neq y\; and\; \phi(x) = \phi(y) = c_i = the\;min\;to\;the\;RLS\\
Consider\;3\;cases:\\
1)\; x,y \in null(A)\land null(L)\\
\Rightarrow since\; null(A)\land null(L) = \{0\}, it\;must\;be\;that\;x = y = 0\\
But\;we\;assumed\; x\neq y\;so\;this\;cannot\;be\;the\;case\\
2)\; x,y \in null(A)\land x,y \not\in null(L)\\
f(x) = ||b||^2 + \lambda||Lx||^2\\
\nabla f(x) = 2\lambda L^TLx
\Rightarrow L^TLx = 0 \rightarrow x \in null(L^TL) \rightarrow x \in null(L)\\
But,\;it\;was\;assumed\;that\;x \not\in null(L)\;so\;this\;case\;cannot\;be\;possible,\;a\;similar\;argument\;can\;be\;made\;for\;y\\
3)\;\; x,y \in null(L)\land x,y \not\in null(A)\\
f(x) = ||Ax-b||^2\\
\nabla f(x) = 2A^TAx - 2A^Tb \rightarrow A^TAx = A^Tb
$

    

    \hypertarget{exercise-6}{%
\section{Exercise 6}\label{exercise-6}}

    \begin{Verbatim}[commandchars=\\\{\}]
{\color{incolor}In [{\color{incolor}14}]:} \PY{n}{x1} \PY{o}{=} \PY{p}{[}\PY{l+m+mi}{0}\PY{p}{,}\PY{l+m+mf}{0.5}\PY{p}{,}\PY{l+m+mi}{1}\PY{p}{,}\PY{l+m+mi}{1}\PY{p}{,}\PY{l+m+mi}{0}\PY{p}{]}
         \PY{n}{x2} \PY{o}{=} \PY{p}{[}\PY{l+m+mi}{0}\PY{p}{,}\PY{l+m+mi}{0}\PY{p}{,}\PY{l+m+mi}{0}\PY{p}{,}\PY{l+m+mi}{1}\PY{p}{,}\PY{l+m+mi}{1}\PY{p}{]}
         \PY{k}{function} \PY{n}{circle\PYZus{}fit}\PY{p}{(}\PY{n}{A}\PY{p}{)}
             \PY{n}{A} \PY{o}{=} \PY{n}{A}\PY{o}{\PYZsq{}}
             \PY{n}{new\PYZus{}A} \PY{o}{=} \PY{l+m+mi}{2}\PY{o}{*}\PY{n}{A}\PY{p}{[}\PY{o}{:}\PY{p}{,}\PY{l+m+mi}{1}\PY{o}{:}\PY{l+m+mi}{1}\PY{p}{]}
             \PY{n}{new\PYZus{}b} \PY{o}{=} \PY{n}{A}\PY{p}{[}\PY{o}{:}\PY{p}{,}\PY{l+m+mi}{1}\PY{o}{:}\PY{l+m+mi}{1}\PY{p}{]}\PY{o}{.\PYZca{}}\PY{l+m+mi}{2}
             \PY{k}{for} \PY{n}{i} \PY{k+kp}{in} \PY{l+m+mi}{2}\PY{o}{:}\PY{n}{size}\PY{p}{(}\PY{n}{A}\PY{p}{,}\PY{l+m+mi}{2}\PY{p}{)}
                 \PY{n}{new\PYZus{}A} \PY{o}{=} \PY{n}{hcat}\PY{p}{(}\PY{n}{new\PYZus{}A}\PY{p}{,} \PY{l+m+mi}{2}\PY{o}{*}\PY{n}{A}\PY{p}{[}\PY{o}{:}\PY{p}{,}\PY{n}{i}\PY{o}{:}\PY{n}{i}\PY{p}{]}\PY{p}{)}
                 \PY{n}{new\PYZus{}b} \PY{o}{=} \PY{n}{new\PYZus{}b} \PY{o}{+} \PY{n}{A}\PY{p}{[}\PY{o}{:}\PY{p}{,}\PY{n}{i}\PY{o}{:}\PY{n}{i}\PY{p}{]}\PY{o}{.\PYZca{}}\PY{l+m+mi}{2}
             \PY{k}{end}
             \PY{n}{new\PYZus{}A} \PY{o}{=} \PY{n}{hcat}\PY{p}{(}\PY{n}{new\PYZus{}A}\PY{p}{,} \PY{n}{ones}\PY{p}{(}\PY{n}{size}\PY{p}{(}\PY{n}{A}\PY{p}{,}\PY{l+m+mi}{1}\PY{p}{)}\PY{p}{)}\PY{p}{)}
             \PY{n}{xls} \PY{o}{=} \PY{o}{*}\PY{p}{(}\PY{n}{inv}\PY{p}{(}\PY{o}{*}\PY{p}{(}\PY{n}{new\PYZus{}A}\PY{o}{\PYZsq{}}\PY{p}{,}\PY{n}{new\PYZus{}A}\PY{p}{)}\PY{p}{)}\PY{p}{,}\PY{o}{*}\PY{p}{(}\PY{n}{new\PYZus{}A}\PY{o}{\PYZsq{}}\PY{p}{,}\PY{n}{new\PYZus{}b}\PY{p}{)}\PY{p}{)}
             \PY{k}{return} \PY{n}{xls}\PY{p}{[}\PY{l+m+mi}{1}\PY{o}{:}\PY{n}{size}\PY{p}{(}\PY{n}{xls}\PY{p}{,}\PY{l+m+mi}{1}\PY{p}{)}\PY{o}{\PYZhy{}}\PY{l+m+mi}{1}\PY{p}{]}\PY{p}{,} \PY{n}{xls}\PY{p}{[}\PY{n}{size}\PY{p}{(}\PY{n}{xls}\PY{p}{,}\PY{l+m+mi}{1}\PY{p}{)}\PY{p}{]}
         \PY{k}{end}
         
         \PY{n}{x}\PY{p}{,} \PY{n}{r} \PY{o}{=} \PY{n}{circle\PYZus{}fit}\PY{p}{(}\PY{n}{vcat}\PY{p}{(}\PY{n}{x1}\PY{o}{\PYZsq{}}\PY{p}{,}\PY{n}{x2}\PY{o}{\PYZsq{}}\PY{p}{)}\PY{p}{)}
\end{Verbatim}


\begin{Verbatim}[commandchars=\\\{\}]
{\color{outcolor}Out[{\color{outcolor}14}]:} ([0.5, 0.541667], -0.08333333333333304)
\end{Verbatim}
            

    % Add a bibliography block to the postdoc
    
    
    
    \end{document}
